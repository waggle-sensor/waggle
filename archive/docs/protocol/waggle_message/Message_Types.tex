\section{Message Types}

\begin{center}
    \rowcolors{1}{rahmen!8}{white}
    \begin{tabular}{ | p{2cm} | p{3cm} | p{4.2cm} | p{5cm} |}
    \hline
    \hline
    \textbf{Major Type} & \textbf{Description} & \textbf{Minor Types} & \textbf{Notes} \\ \hline \hline
    $0x72$ (`r') & Registration & $0x69$ (`i'), $0x75$ (`u'), $0x72$ (`r'), $0x6e$ (`n'), $0x64$ (`d'), $0x61$ (`a') & This message type will evolve to offer 
several registration based 
functions \\ 
\hline
    $0x70$ (`p') & Heartbeat / Ping   & $0x72$ (`r'), $0x61$ (`a') &   \\ \hline
    $0x61$ (`a') & Acknowledgement & Any  &   \\ \hline
    $0x74$ (`t') & Time   & $0x72$ (`r'), $0x61$ (`a'), $0x75$ (`u') &   \\ \hline
    $0x73$ (`s') & Sensor Data   &  $0x64$ (`d') &   \\ \hline
    $0x6c$ (`l') & Location   &  $0x72$ (`r'), $0x6d$ (`m'), $0x65$ (`e'), $0x6c$ (`l'), $0x80$, $0x81$, $0x82$, $0x83$, $0x90$, $0x91$, $0x92$, $0x93$, $0xa0$, 
$0xa1$ ,$0xa2$ ,$0xa3$ ,$0xb0$ ,$0xb1$ ,$0xb2$ ,$0xb3$  &   \\ \hline
    $0x64$ (`d') & Command / Response   & $0x63$ (`c'), $0x72$ (`r'), $0x6f$ (`o'), $0x73$ (`s') &   \\ \hline
    $0x63$ (`c') & Combined Message   & $0x66$ (`f') &   \\ \hline
    \end{tabular}
\end{center}

\newpage

\subsection{Registration}
\subsubsection{Initial Registration Message --- Deprecated}
\textbf{Header:}
\begin{center}
    \rowcolors{1}{rahmen!8}{white}
    \begin{tabular}{ | l | l | p{3cm} | p{5cm} |}
    \hline
    \hline
    \textbf{Byte Field} & \textbf{Field Name} & \textbf{Value} & \textbf{Notes} \\ \hline \hline
    8 & Message Major Type & $0x72$ (`r') & This is normally the first message sent by a new instance. \\    \hline
    9 & Message Minor Type & $0x69$ (`i') &  \\    \hline
    23--27 & Reference Sequence ID & $0xzz$ $0xzz$ $0xzz$ & Present if sent as a response to `rr' message.\\ \hline
    \end{tabular}
\end{center}
\noindent
\textbf{Payload:}
\begin{framed}
In this protocol version, we will use the current registration message format in the body.
\end{framed}


\subsubsection{Registration Update --- Not Implemented}
\textbf{Header:}
\begin{center}
    \rowcolors{1}{rahmen!8}{white}
    \begin{tabular}{ | l | l | p{3cm} | p{5cm} |}
    \hline
    \hline
    \textbf{Byte Field} & \textbf{Field Name} & \textbf{Value} & \textbf{Notes} \\ \hline \hline
    8 & Message Major Type & $0x72$ (`r') & A node can update its Registration with this message. \\    \hline
    9 & Message Minor Type & $0x75$ (`u') &  \\    \hline
    \end{tabular}
\end{center}
\noindent
\textbf{Payload:}
\begin{framed}
In this protocol version, both initial and updated registrations will contain the full
registration message. The update only alerts the cloud to look for an existing registration.
The cloud may update or overwrite the old registration.
\end{framed}

\subsubsection{Request for Registration}
\textbf{Header:}
\begin{center}
    \rowcolors{1}{rahmen!8}{white}
    \begin{tabular}{ | l | l | p{3cm} | p{5cm} |}
    \hline
    \hline
    \textbf{Byte Field} & \textbf{Field Name} & \textbf{Value} & \textbf{Notes} \\ \hline \hline
    8 & Message Major Type & $0x72$ (`r') &  \\    \hline
    9 & Message Minor Type & $0x72$ (`r') &  \\    \hline
    \end{tabular}
\end{center}
\noindent
\textbf{Payload:}
\begin{framed}
Meta data of the requestor (e.g., name, version, instance, etc.).
\end{framed}

\subsubsection{Request for Configuration Registration}
\textbf{Header:}
\begin{center}
    \rowcolors{1}{rahmen!8}{white}
    \begin{tabular}{ | l | l | p{3cm} | p{5cm} |}
    \hline
    \hline
    \textbf{Byte Field} & \textbf{Field Name} & \textbf{Value} & \textbf{Notes} \\ \hline \hline
    8 & Message Major Type & $0x72$ (`r') &  \\    \hline
    9 & Message Minor Type & $0x6e$ (`n') &  \\    \hline
    \end{tabular}
\end{center}
\noindent
\textbf{Payload:}
\begin{framed}
Configuration information of the sender.
\end{framed}

\subsubsection{Request for De-registration}
\textbf{Header:}
\begin{center}
    \rowcolors{1}{rahmen!8}{white}
    \begin{tabular}{ | l | l | p{3cm} | p{5cm} |}
    \hline
    \hline
    \textbf{Byte Field} & \textbf{Field Name} & \textbf{Value} & \textbf{Notes} \\ \hline \hline
    8 & Message Major Type & $0x72$ (`r') &  \\    \hline
    9 & Message Minor Type & $0x64$ (`d') &  \\    \hline
    \zerolenmessage
    \end{tabular}
\end{center}
\noindent
\textbf{Payload:}
\begin{framed}
None.
\end{framed}

\subsubsection{Registration Response}
\textbf{Header:}
\begin{center}
    \rowcolors{1}{rahmen!8}{white}
    \begin{tabular}{ | l | l | p{3cm} | p{5cm} |}
    \hline
    \hline
    \textbf{Byte Field} & \textbf{Field Name} & \textbf{Value} & \textbf{Notes} \\ \hline \hline
    8 & Message Major Type & $0x72$ (`r') &  \\    \hline
    9 & Message Minor Type & $0x61$ (`a') &  \\    \hline
    \end{tabular}
\end{center}
\noindent
\textbf{Payload:}
\begin{framed}
Responder's message.
\end{framed}

\subsection{Alive Heartbeat}
\subsubsection{Onetime Heartbeat Ping Request}
\textbf{Header:}
\begin{center}
    \rowcolors{1}{rahmen!8}{white}
    \begin{tabular}{ | l | l | p{3cm} | p{5cm} |}
    \hline
    \hline
    \textbf{Byte Field} & \textbf{Field Name} & \textbf{Value} & \textbf{Notes} \\ \hline \hline
    8 & Message Major Type & $0x70$ (`p') &  \\    \hline
    9 & Message Minor Type & $0x72$ (`r') &  \\    \hline
    \zerolenmessage
    \end{tabular}
\end{center}
\noindent
\textbf{Payload:}
\begin{framed}
None.
\end{framed}


\subsubsection{Onetime Heartbeat Pong}
\textbf{Header:}
\begin{center}
    \rowcolors{1}{rahmen!8}{white}
    \begin{tabular}{ | l | l | p{3cm} | p{5cm} |}
    \hline
    \hline
    \textbf{Byte Field} & \textbf{Field Name} & \textbf{Value} & \textbf{Notes} \\ \hline \hline
    8 & Message Major Type & $0x70$ (`p') &  \\    \hline
    9 & Message Minor Type & $0x61$ (`a') &  \\    \hline
    2--3 & Length of Message Body & $0x00$ $0x04$ & 4 Byte message. \\ \hline
    \end{tabular}
\end{center}
\noindent
\textbf{Payload:}
\begin{framed}
``Pong''
\end{framed}


\subsection{Acknowledgement}
\subsubsection{Message Receipt -- Not Implemented}
\textbf{Header:}
\begin{center}
    \rowcolors{1}{rahmen!8}{white}
    \begin{tabular}{ | l | l | p{3cm} | p{5cm} |}
    \hline
    \hline
    \textbf{Byte Field} & \textbf{Field Name} & \textbf{Value} & \textbf{Notes} \\ \hline \hline
    8 & Message Major Type & $0x61$ (`a') &  \\    \hline
    9 & Message Minor Type & $0xzz$ & Type of the message being acknowledged. \\    \hline
    \zerolenmessage
    35--37 & Reference Sequence ID & $0xzz$ $0xzz$ $0xzz$ & Sequence number of the message being acknowledged.\\ \hline

    \end{tabular}
\end{center}
\noindent
\textbf{Payload:}
\begin{framed}
None.
\end{framed}

\subsubsection{Completion of Task}
This message will change on a case by case basis. I am not sure if we will have to classify this under the
acknowledgement case, but I think we should have some way of Acknowledging that.

\subsection{Time}
\url{http://tools.ietf.org/html/rfc958}
\subsubsection{Request Current Time}
\textbf{Header:}
\begin{center}
    \rowcolors{1}{rahmen!8}{white}
    \begin{tabular}{ | l | l | p{3cm} | p{5cm} |}
    \hline
    \hline
    \textbf{Byte Field} & \textbf{Field Name} & \textbf{Value} & \textbf{Notes} \\ \hline \hline
    8 & Message Major Type & $0x74$ (`t') &  \\    \hline
    9 & Message Minor Type & $0x72$ (`r') & \\    \hline
    \zerolenmessage
    \end{tabular}
\end{center}
\noindent
\textbf{Payload:}
\begin{framed}
None.
\end{framed}

\subsubsection{Current Time Response}
\textbf{Header:}
\begin{center}
    \rowcolors{1}{rahmen!8}{white}
    \begin{tabular}{ | l | l | p{3cm} | p{5cm} |}
    \hline
    \hline
    \textbf{Byte Field} & \textbf{Field Name} & \textbf{Value} & \textbf{Notes} \\ \hline \hline
    8 & Message Major Type & $0x74$ (`t') &  \\    \hline
    9 & Message Minor Type & $0x61$ (`a') & \\    \hline
    17--19 & Length of Message Body & $0x00$ $0xzz$ & N--bytes message. \\ \hline
    \end{tabular}
\end{center}
\noindent
\textbf{Payload:}
\begin{framed}
A floatting point number of the current time since the epoch.
\end{framed}

\subsubsection{Time Update -- Not Implemented}
\textbf{Header:}
\begin{center}
    \rowcolors{1}{rahmen!8}{white}
    \begin{tabular}{ | l | l | p{3cm} | p{5cm} |}
    \hline
    \hline
    \textbf{Byte Field} & \textbf{Field Name} & \textbf{Value} & \textbf{Notes} \\ \hline \hline
    23 & Message Major Type & $0x74$ (`t') &  \\    \hline
    24 & Message Minor Type & $0x75$ (`u') & \\    \hline
    17--19 & Length of Message Body & $0x00$ $0x00$ $0x04$ & 4 Byte message. \\ \hline
    \end{tabular}
\end{center}
\noindent
\textbf{Payload:}
\begin{framed}
\vskip 0.25 in
\begin{center}
\begin{bytefield}[bitwidth=1.5in]{4}
\bitheader{0,1,2,3} \\
\bitbox{4}{Time 4 (0), Time 3 (1), Time 2 (2), Time 1 (3)}\\
\end{bytefield}
\end{center}
\end{framed}

\begin{framed}
\textbf{Time Representation:}
\\
Current Epoch Time\_Sec = (Time 4 ${<<}$ 24) + (Time 3 ${<<}$ 16) + (Time 2 ${<<}$ 8) + (Time 1)
\end{framed}

\subsection{Sensor Data}
The data is serialized and compressed before sending.

When size of data exceeds the maximum limit of data length (or limit that system designer set) multiple Waggle messages can be used to send the data. For this 
case, both $Ext\_header$ and $MMSG$ in $Optional\_key$ fields will be set. In addition, there are 6 bytes of information added in the message body of each 
message (See parameter description in \ref{parameter_description}). $Snd\_Seq$ will be the same for all the sub messages.
\\
\textbf{Header:}
\begin{center}
    \rowcolors{1}{rahmen!8}{white}
    \begin{tabular}{ | l | l | p{3cm} | p{5cm} |}
    \hline
    \hline
    \textbf{Byte Field} & \textbf{Field Name} & \textbf{Value} & \textbf{Notes} \\ \hline \hline
    8 & Message Major Type & $0x73$ (`s') &  \\    \hline
    9 & Message Minor Type & $0x64$ (`d') & \\    \hline
    \end{tabular}
\end{center}

\noindent
\textbf{Payload:}

\begin{framed}
Serialized data that are compressed.
\end{framed}

\subsection{Location}
\url{http://dev.w3.org/geo/api/spec-source.html#api_description}\\
\url{http://resources.arcgis.com/en/help/main/10.1/index.html#//009t0000023w000000}\\


\subsubsection{Request Current Location -- Not Implemented}
\textbf{Header:}
\begin{center}
    \rowcolors{1}{rahmen!8}{white}
    \begin{tabular}{ | l | l | p{3cm} | p{5cm} |}
    \hline
    \hline
    \textbf{Byte Field} & \textbf{Field Name} & \textbf{Value} & \textbf{Notes} \\ \hline \hline
    8 & Message Major Type & $0x6c$ (`l') &  \\    \hline
    9 & Message Minor Type & $0x72$ (`r') & \\    \hline
    \zerolenmessage
    \end{tabular}
\end{center}
\noindent
\textbf{Payload:}
\begin{framed}
None.
\end{framed}


\subsubsection{Location Standardized -- Not Implemented}

\begin{framed}
We will use WGS84 geodetic datum.
\begin{itemize}
\item \textbf{Latitude:} Latitude and Longitude of point. Northern latitudes are positive, southern latitudes are negative.
  eastern longitudes are positive, western longitudes are negative. Valid formats include (Max 16 allowed):
\begin{itemize}
  \item N43°38'19.39", W116°14'28.86" - \textbf{LatLon Type:} ${0x00}$
  \item 43°38'19.39"N, 116°14'28.86"W - \textbf{LatLon Type:} ${0x01}$
  \item 43 38 19.39, -116 14 28.86 - \textbf{LatLon Type:} ${0x02}$
  \item 43.63871944444445, -116.2413513485235 - \textbf{LatLon Type:} ${0x03}$

\end{itemize}
\item \textbf{Altitude:} Elevation of the point. Valid formats include (Max 8 allowed):
\begin{itemize}
   \item Orthometric in Meters - \textbf{Elevation Type:} ${0x00}$
   \item Geoid in Meters - \textbf{Elevation Type:} ${0x01}$
   \item Ellipsoidal in Meters - \textbf{Elevation Type:} ${0x02}$
   \item Non-standard (10 m above pedestal) in Meters - \textbf{Elevation Type:} ${0x03}$
\end{itemize}

\end{itemize}

There are 16 different formats for representation of position, which are derived using a
combination of the 4 LatLon and 4 Elevation types. The minor type uses the upper nibble to
state the Latitude and Longitude type and the lower nibble for the Elevation type.
\\

\center{$Message\ Minor\ Type\ =\ {0x80\ |\ Elevation\ Type}\ << 4\ |\ {LatLon\ Type} $}

\end{framed}
\textbf{Header:}
\begin{center}
    \rowcolors{1}{rahmen!8}{white}
    \begin{tabular}{ | l | l | p{3cm} | p{5cm} |}
    \hline
    \hline
    \textbf{Byte Field} & \textbf{Field Name} & \textbf{Value} & \textbf{Notes} \\ \hline \hline
    8 & Message Major Type & $0x6c$ (`l') &  \\    \hline
    9 & Message Minor Type & $Generated$ & 16 different representations.  \\   \hline
    \end{tabular}
\end{center}
\noindent
\textbf{Payload:}
\begin{framed}
We will have a delimited payload of the following kind -
\\
$Latitude\_[0]\_Longitude\_[0]\_Elevation$
\end{framed}

\subsubsection{Location Meta -- Not Implemented}
\textbf{Header:}
\begin{center}
    \rowcolors{1}{rahmen!8}{white}
    \begin{tabular}{ | l | l | p{3cm} | p{5cm} |}
    \hline
    \hline
    \textbf{Byte Field} & \textbf{Field Name} & \textbf{Value} & \textbf{Notes} \\ \hline \hline
    8 & Message Major Type & $0x6c$ (`l') &  \\    \hline
    9 & Message Minor Type & $0x6d$ (`m') & Meta Data of the location. \\   \hline
    \end{tabular}
\end{center}
\noindent
\textbf{Payload:}
\begin{framed}
The payload of the message will be considered as a text string describing the location.
This is for human readable description like - Under a tree, In the shade, on the top of
the HVAC unit and so on.
\end{framed}


\subsubsection{Get Location Estimator Type -- Not Implemented}
\textbf{Header:}
\begin{center}
    \rowcolors{1}{rahmen!8}{white}
    \begin{tabular}{ | l | l | p{3cm} | p{5cm} |}
    \hline
    \hline
    \textbf{Byte Field} & \textbf{Field Name} & \textbf{Value} & \textbf{Notes} \\ \hline \hline
    8 & Message Major Type & $0x6c$ (`l') &  \\    \hline
    9 & Message Minor Type & $0x65$ (`e') & \\    \hline
    \zerolenmessage
    \end{tabular}
\end{center}
\noindent
\textbf{Payload:}
\begin{framed}
None.
\end{framed}

\subsubsection{Location Estimator Type -- Not Implemented}

\begin{framed}
This message will both provide location estimator type and also the error bounds for
the Latitude, Longitude and Elevation.
\\
\\
Estimator Types:
\begin{itemize}

\item Preset Static - \textbf{Position Type:} $0x00$
\item Satellite - \textbf{Position Type:} $0x01$
\item Dead Reckoning or Software Estimation - \textbf{Position Type:} $0x02$
\item Ranging - \textbf{Position Type:} $0x03$

\end{itemize}
Errors Types:
\begin{itemize}
\item Linear Error (LE90) in meters - \textbf{Error Type:} $0x00$
\item Circular Error (CE90) in meters - \textbf{Error Type:} $0x01$
\end{itemize}

\center{$Estimation\ Type\ =\ Position\ Type\ <<\ 4\ |\ (Error\ Type)$}

\end{framed}

\textbf{Header:}

\begin{center}
    \rowcolors{1}{rahmen!8}{white}
    \begin{tabular}{ | l | l | p{3cm} | p{5cm} |}
    \hline
    \hline
    \textbf{Byte Field} & \textbf{Field Name} & \textbf{Value} & \textbf{Notes} \\ \hline \hline
    8 & Message Major Type & $0x6c$ (`l') &  \\    \hline
    9 & Message Minor Type & $0x6c$ (`l') &  \\    \hline
    \end{tabular}
\end{center}
\vskip 0.1in
\noindent
\textbf{Payload:}
\begin{framed}
Delimited payload -
\\
$Latitude\_Estimation\_Type\_[1]\_Latitude\_Error\_[0]\_\\
Longitude\_Estimation\_Type\_[1]\_Longitude\_Error\\
\_[0]\_Elevation\_Estimation\_Type\_[1]\_Elevation\_Error$
\end{framed}


\subsubsection{Set Location Interrupt}
\begin{framed}
\textit{The following will depend on the Location Engine.
What are the features and facilities we want to include in this
engine? Can we set location alerts? i.e. alert me when you have
reached location x,y,z, alert me when you have moved further
than 10 ft from base (a floating platform?). May not
be a core waggle feature, but something the particular
instance will implement.}
\end{framed}
\subsubsection{Get Current Location Interrupt}
\begin{framed}
\textit{The following will depend on the Location Engine.
What are the features and facilities we want to include in this
engine? Can we set location alerts? i.e. alert me when you have
reached location x,y,z, alert me when you have moved further
than 10 ft from base (a floating platform?). May not be a
core waggle feature, but something the particular
instance will implement.}
\end{framed}

\subsection{Direct Command Message}

\textbf{Header:}
\begin{center}
    \rowcolors{1}{rahmen!8}{white}
    \begin{tabular}{ | l | l | p{3cm} | p{5cm} |}
    \hline
    \hline
    \textbf{Byte Field} & \textbf{Field Name} & \textbf{Value} & \textbf{Notes} \\ \hline \hline
    8 & Message Major Type & $0x64$ (`d') &  \\    \hline
    9 & Message Minor Type & $0x63$ (`c') &  \\    \hline
    \end{tabular}
\end{center}
\noindent
\textbf{Payload:}
\begin{framed}
Specific Payload
\end{framed}


\subsection{Direct Request Message}

\textbf{Header:}
\begin{center}
    \rowcolors{1}{rahmen!8}{white}
    \begin{tabular}{ | l | l | p{3cm} | p{5cm} |}
    \hline
    \hline
    \textbf{Byte Field} & \textbf{Field Name} & \textbf{Value} & \textbf{Notes} \\ \hline \hline
    8 & Message Major Type & $0x64$ (`d') &  \\    \hline
    9 & Message Minor Type & $0x72$ (`r') &  \\    \hline
    \end{tabular}
\end{center}
\noindent
\textbf{Payload:}
\begin{framed}
Specific payload.
\end{framed}

\subsection{Direct OS Specific Request Message}

\textbf{Header:}
\begin{center}
    \rowcolors{1}{rahmen!8}{white}
    \begin{tabular}{ | l | l | p{3cm} | p{5cm} |}
    \hline
    \hline
    \textbf{Byte Field} & \textbf{Field Name} & \textbf{Value} & \textbf{Notes} \\ \hline \hline
    8 & Message Major Type & $0x64$ (`d') &  \\    \hline
    9 & Message Minor Type & $0x6f$ (`o') &  \\    \hline
    \end{tabular}
\end{center}
\noindent
\textbf{Payload:}
\begin{framed}
Specific payload.
\end{framed}

\subsection{Direct Shell Command}

\textbf{Header:}
\begin{center}
    \rowcolors{1}{rahmen!8}{white}
    \begin{tabular}{ | l | l | p{3cm} | p{5cm} |}
    \hline
    \hline
    \textbf{Byte Field} & \textbf{Field Name} & \textbf{Value} & \textbf{Notes} \\ \hline \hline
    8 & Message Major Type & $0x64$ (`d') &  \\    \hline
    9 & Message Minor Type & $0x73$ (`s') &  \\    \hline
    \end{tabular}
\end{center}
\noindent
\textbf{Payload:}
\begin{framed}
Specific payload.
\end{framed}


\begin{landscape}

\subsection{Combined Message -- Needs revision...}
\begin{framed}
\begin{itemize}
\item Packet Encapsulation
\begin{itemize}

\item Check individual messages being forwarded together for integrity.
\item Arrange individual messages one after another in the forwarding message body.
\item Compute packet CRC32.

\end{itemize}

\item Packet Decoding
\begin{itemize}
\item Check Header and Message CRC.
\item Read first 32 bytes in body for the header of the first encapsulated message.
\item Extract first message body and CRC32.
\item Process the encapsulated message as a regular message.
\item Continue until the end of all encapsulated messages.
\end{itemize}
\end{itemize}
\end{framed}

\begin{center}
    \rowcolors{1}{rahmen!8}{white}
    \begin{tabular}{ | l | l | p{3cm} | p{5cm} |}
    \hline
    \hline
    \textbf{Byte Field} & \textbf{Field Name} & \textbf{Value} & \textbf{Notes} \\ \hline \hline
    8 & Message Major Type & $0x63$ (`c') &  \\    \hline
    9 & Message Minor Type & $0x66$ (`f') & \\    \hline
    \end{tabular}
\end{center}
\begin{bytefield}[bitwidth=2.2in]{4}
\bitheader{0,1,2,3} \\
\begin{rightwordgroup}{40 Byte\\ Packet\\ Header}
\bitbox{1}{Prot\_Ver: Maj\_N:Min\_N (0)}
\bitbox{1}{Flag: Dev\_P:Msg\_P:Pref (1)}
\bitbox{2}{Length of Message Body: Len\_byte 2 (2), Len\_byte 1 (3)}
\\
\bitbox{4}{Message Time Stamp: Time 4 (4), Time 3 (5), Time 2 (6), Time 1 (7)}
\\
\bitbox{1}{Msg\_Mj\_Type (8)}
\bitbox{1}{Msg\_Mi\_Type (9)}
\bitbox{1}{Ext\_Header (10) }
\bitbox{1}{Optional Key (11) }
\\
\bitbox{4}{S\_UniqID\_byte 8 (12), S\_UniqID\_byte 7 (13), S\_UniqID\_byte 6 (14), S\_UniqID\_byte 5 (15)}
\\
\bitbox{4}{S\_UniqID\_byte 4 (16), S\_UniqID\_byte 3 (17), S\_UniqID\_byte 2 (18), S\_UniqID\_byte 1 (19)}
\\
\bitbox{4}{R\_UniqID\_byte 8 (20), R\_UniqID\_byte 7 (21), R\_UniqID\_byte 
6 (22), R\_UniqID\_byte 5 (23)}
\\
\bitbox{4}{R\_UniqID\_byte 4 (24), R\_UniqID\_byte 3 (25), R\_UniqID\_byte 2 
(26), R\_UniqID\_byte 1 (27)}
\\
\bitbox{2}{Snd Session Number: Session\_No\_Hi (28), Session\_No\_Lo (29)}
\bitbox{2}{Resp Session Number: Session\_No\_Hi (30), Session\_No\_Lo (31)}
\\
\bitbox{3}{Snd\_Seq 3 (32), Snd\_Seq 2 (33), Snd\_Seq 1 (34)}
\bitbox{1}{Resp\_Seq 3 (35)}
\\
\bitbox{2}{Resp\_Seq 2 (36), Resp\_Seq 1 (37)}
\bitbox{2}{CRC\_16\_byte1 (38), CRC\_16\_byte2 (39)}
\end{rightwordgroup}
\\
\begin{rightwordgroup}{N Byte\\ Message\\ Payload}
\\
\bitbox{4}{[Header1][Body1][Footer1]}
\\
\bitbox{4}{[Header2][Body2][Footer2]}
\\
\bitbox{4}{\vdots }
\\
\bitbox{4}{...}
\end{rightwordgroup}
\\
\\
\begin{rightwordgroup}{4 Byte\\ Packet\\ Footer}
\bitbox{4}{CRC\_32 (39+Len(N--Bytes Data))}
\end{rightwordgroup}
\end{bytefield}
\end{landscape}
