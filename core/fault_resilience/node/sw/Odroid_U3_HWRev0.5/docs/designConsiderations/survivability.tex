\section{Survivability}

Why is survivability required

Expensive to physically access the devices (huge numbers, transportation cost, etc)

\subsubsection{Detection of Improper Shutdown}

Detect improper shutdown of a system so as to run self-check and reconfigure if need be.

This can be implemented through a script that is executed during boot and shutdown. An example for such a scipt, based on the upstart framework, is given below.

\lstinputlisting{./data/improper_shutdown.conf}

\subsection{Redundancy in Kernel Boot Process}

Create conditional booting based on hash checks of binaries.

\subsection{Periodic Check of File System Consistency}

File system consistency needs to be checked periodicly and at specific key steps (before shutdown, during boot). Find hardware problems or file system inconsistent state. Use of journaling file system where possible.

\subsection{Graceful Degradation}
Graceful degradation is an important aspect of survivability. If left unconstrained faulty parts can result in a useless board, whereas, taming and instrumentation can prolong its lifespan and continue of operation until the next maintenance/repair/replace cycle.

\subsubsection{Improper Shutdown Detection}

An improper shutdown can be a sign that the systems resources need to be restricted as they make it unstable.

\begin{itemize}
 \item Crash due to overheating: reduce number of active CPUs
 \item Crash due to power supply problem (IR drop): switch unnecessary modules off
 \item Memory error: isolate memory bank
 \item Kernel crash: disable faulty module
 \item File system error: disable sector
 \item Run out of power: try to improve power usage (coordination of battery and renewable resources)
\end{itemize}




\subsubsection{Reduce number of available CPUs and Memory Nodes}

A part of graceful degradation would be the choice to restrict the number of active (i.e. available for execution) CPUs of the system, or the setting of an upper utilization limit for the processing cores. For this we can use the ``cpuset'' application. See \url{http://man7.org/linux/man-pages/man7/cpuset.7.html}, \url{https://www.kernel.org/doc/Documentation/cgroups/cpusets.txt}, \url{https://www.suse.com/documentation/slerte_11/slerte_tutorial/data/slerte_tutorial.html}, \url{http://techpubs.sgi.com/library/tpl/cgi-bin/getdoc.cgi?coll=linux&db=bks&srch=&fname=/SGI_Admin/LX_Resource_AG/sgi_html/ch01.html} and \url{https://code.google.com/p/cpuset/}

\subsection{System Monitoring}

Logging of system metrics such as utilization, cpu frequency changes, network traffic, logins, etc... can be useful for security, debugging and graceful degradation.

See appendix A for monit. See appendix B for collectl and colmux.

\subsection{Kernel Panic and Module Crash Handling}

Kernel panics are impossible to contain and would certainly result in a system freeze, in such a case the system needs record as much state as possible to help debugging and then force a restart possible using different kernel version. Module crashes are more managable and will not necesarily leave the system hanging, as in the kernel case state needs to be logged and an attempt to clean and reload the module needs to be made. If a reload is impossible and the module serves an important function a full system restart may be used.

\subsection{Application Restart}

In case of crashes important applications need to be restarted. Use of 'monit' or 'systemd'
