\documentclass[a4paper,10pt]{article}
\usepackage[utf8]{inputenc}
\usepackage{hyperref}
\usepackage{listings}
\usepackage{graphicx}
\usepackage{color}
\usepackage{multirow}

\newcommand{\tobetested}[0]{\colorbox{red}{\textbf{To Be Tested}}}
\newcommand{\tested}[0]{\colorbox{green}{\textbf{Working}}}

%opening
\title{Design considerations for embedded systems with sensing capabilities in urban areas}
\author{}

\begin{document}

\maketitle

\begin{abstract}
 The following document describes the design considerations when building an embedded system that has sensing capabilities and is to be deployment in an urban area.
\end{abstract}

\newpage

\tableofcontents

\newpage

\lstset{ %
%  backgroundcolor=\color{white},   % choose the background color; you must add \usepackage{color} or \usepackage{xcolor}
  basicstyle=\small\ttfamily,        % the size of the fonts that are used for the code
%  breakatwhitespace=false,         % sets if automatic breaks should only happen at whitespace
  breaklines=true,                 % sets automatic line breaking
  captionpos=t,                    % sets the caption-position to bottom
  commentstyle=\color{green},    % comment style
%  deletekeywords={...},            % if you want to delete keywords from the given language
  escapeinside={\%*}{*)},          % if you want to add LaTeX within your code
%  extendedchars=true,              % lets you use non-ASCII characters; for 8-bits encodings only, does not work with UTF-8
  frame=single,                    % adds a frame around the code
%  keepspaces=true,                 % keeps spaces in text, useful for keeping indentation of code (possibly needs columns=flexible)
%  keywordstyle=\color{blue},       % keyword style
%  language=Octave,                 % the language of the code
%  morekeywords={*,...},            % if you want to add more keywords to the set
  numbers=left,                    % where to put the line-numbers; possible values are (none, left, right)
  numbersep=5pt,                   % how far the line-numbers are from the code
  numberstyle=\tiny\color{red}, % the style that is used for the line-numbers
%  rulecolor=\color{black},         % if not set, the frame-color may be changed on line-breaks within not-black text (e.g. comments (green here))
%  showspaces=false,                % show spaces everywhere adding particular underscores; it overrides 'showstringspaces'
%  showstringspaces=false,          % underline spaces within strings only
%  showtabs=false,                  % show tabs within strings adding particular underscores
  stepnumber=1,                    % the step between two line-numbers. If it's 1, each line will be numbered
%  stringstyle=\color{mymauve},     % string literal style
%  tabsize=2,                       % sets default tabsize to 2 spaces
  title=\lstname                   % show the filename of files included with \lstinputlisting; also try caption instead of title
}


\section{Introduction}

Why is it different to deploy sensors in an urban environment than deploying them on a rural setting. The main difference lies on the presence of people and their right to privacy. Secondary differences lie on the easier access to power, increased chance of human activity interference to communication (in case it is wireless).

Sensors can measure a variety on metrics. Some are not consider privacy intrusive such as temperature, humidity or wind speed. Others can be considered intrusive based on their usage, post processing, in combination with other metrics or aggregation from different locations. Such potentially intrusive sensing data are capture of image, sound, video, wireless traffic or rfid card presence.

Based on the above, a list of design considerations can be constructed.


\paragraph{First order design considerations}

\begin{itemize}
 \item Privacy: Protect the privacy of people that come in sensing vicinity from the embedded devices
 \item Security: Preserve the integrity of the system, the data and assure no privacy breach
 \item Survivability: Survive hardware or software failures due to bugs, weather conditions, or natural wear-out.
 \item Power Efficiency: Minimize amount of required operational power, stay inside provided energy budget, and instrument power source (battery, renewable energy) to increase availability.
\end{itemize}

\paragraph{Second order design considerations}

\begin{itemize}
 \item Network Communication Efficiency: Minimize network traffic 
 \item Cost: Hardware, software, and maintenance cost
\end{itemize}

Following we will investigate different design choices that try to improve the above goals in the best possible way. The assumed scheme would be WAggLE (see below) and the deployment location the city of Chicago.

\subsection{The WAggLE architecture}

\subsubsection{Cloud}

From the Node Controller side it is where data are send and commands are received from.
From the external user side it is where commands are send to so that they are forwarded to the nodes and also the public access point of sensor data
From the Guest Node side the cloud is unreachable

\subsubsection{Node Controller}

From the external user side it is not accessible
From the cloud side it is the source of sensor data and the destination for commands
From the guest node side it is the destination of collected data and a liaison for requesting data of other sensors


\subsubsection{Guest Node}

From the cloud and external use side the Guest Node is unreachable
From the Node Controller side it is the source of sensor data and destination of requested data.


\subsection{City of Chicago}

\url{https://en.wikipedia.org/wiki/Climate_of_Chicago}

Low temperatures on most of the year.

Can get quite hot during summer.

High humidity.

Rainfall.

Snow.

Stray Bullets.

\section{Privacy}

As mention earlier privacy becomes a first order considerations when deploying sensors on an urban environment. The reason for that is that as the sensors collect data, sensitive information about habitats or travelers around them may also be collected. This can happen accidentally or with a malicious intent. Following a list of possible sensors together with normal and abusive scenarios is gonna be presented.

The case of a wireless cards: a wireless card can act as an activity/traffic sensor. Metrics that can be recorded include traffic load, unique MAC addresses, possibly unique urls of unencrypted communication, or even keyword occurrence of unencrypted instant messaging. These information can be used to present an image of how communication heavy the area is, or how event news propagate from their source. However, miss-use can also happen and personal information can be extracted; like personal messages, preferences, or even walking routes and presence inside building.
  
The case of RFID readers: An RFID reader can be used for enabling some feature on the sensor or communication of the sensor and an application in an approaching device. In a malicious scenario the rfid reader can be used to extract information from rfid cards that come into its vicinity and otherwise have no correlation with the system (Ventra Cards, ID Cards, Credit Cards).
  
The case of image or video recording: image capturing can be used in conjunction with a figure recognition/tracking software to present information about the density of people or vehicles near the node. A possible miss-use would be face recognition for route tracking, presence verification, or even extraction of conversations.
  
The case of audio recording: audio recording can be used to collect information about noise levels or species identification in a location. Or it can be used to record conversations.
  
  
For the aforementioned reasons requirements about the operation and oversight of the sensors' data collection need to be specified.

\subsection{Operation Requirements}
Should we require some kind of key for a sensor to join the embedded system?


The embedded system should have control over what it senses all the time.
What is sensed should be public at all time and monitored so that in case of a privacy incident information leakage is minimized

Automatic isolation/lock of offending sensors.

Particularly intrusive sensors may need to be left out and alternatives should be considered.



\subsection{Oversight Requirements}

An independent, from the operational authority, observer should be able to have a full picture of what has been collected by the sensors and who has access to it in an easy manner.

\input{security}
\section{Survivability}

Why is survivability required

Expensive to physically access the devices (huge numbers, transportation cost, etc)

\subsubsection{Detection of Improper Shutdown}

Detect improper shutdown of a system so as to run self-check and reconfigure if need be.

This can be implemented through a script that is executed during boot and shutdown. An example for such a scipt, based on the upstart framework, is given below.

\lstinputlisting{./data/improper_shutdown.conf}

\subsection{Redundancy in Kernel Boot Process}

Create conditional booting based on hash checks of binaries.

\subsection{Periodic Check of File System Consistency}

File system consistency needs to be checked periodicly and at specific key steps (before shutdown, during boot). Find hardware problems or file system inconsistent state. Use of journaling file system where possible.

\subsection{Graceful Degradation}
Graceful degradation is an important aspect of survivability. If left unconstrained faulty parts can result in a useless board, whereas, taming and instrumentation can prolong its lifespan and continue of operation until the next maintenance/repair/replace cycle.

\subsubsection{Improper Shutdown Detection}

An improper shutdown can be a sign that the systems resources need to be restricted as they make it unstable.

\begin{itemize}
 \item Crash due to overheating: reduce number of active CPUs
 \item Crash due to power supply problem (IR drop): switch unnecessary modules off
 \item Memory error: isolate memory bank
 \item Kernel crash: disable faulty module
 \item File system error: disable sector
 \item Run out of power: try to improve power usage (coordination of battery and renewable resources)
\end{itemize}




\subsubsection{Reduce number of available CPUs and Memory Nodes}

A part of graceful degradation would be the choice to restrict the number of active (i.e. available for execution) CPUs of the system, or the setting of an upper utilization limit for the processing cores. For this we can use the ``cpuset'' application. See \url{http://man7.org/linux/man-pages/man7/cpuset.7.html}, \url{https://www.kernel.org/doc/Documentation/cgroups/cpusets.txt}, \url{https://www.suse.com/documentation/slerte_11/slerte_tutorial/data/slerte_tutorial.html}, \url{http://techpubs.sgi.com/library/tpl/cgi-bin/getdoc.cgi?coll=linux&db=bks&srch=&fname=/SGI_Admin/LX_Resource_AG/sgi_html/ch01.html} and \url{https://code.google.com/p/cpuset/}

\subsection{System Monitoring}

Logging of system metrics such as utilization, cpu frequency changes, network traffic, logins, etc... can be useful for security, debugging and graceful degradation.

See appendix A for monit. See appendix B for collectl and colmux.

\subsection{Kernel Panic and Module Crash Handling}

Kernel panics are impossible to contain and would certainly result in a system freeze, in such a case the system needs record as much state as possible to help debugging and then force a restart possible using different kernel version. Module crashes are more managable and will not necesarily leave the system hanging, as in the kernel case state needs to be logged and an attempt to clean and reload the module needs to be made. If a reload is impossible and the module serves an important function a full system restart may be used.

\subsection{Application Restart}

In case of crashes important applications need to be restarted. Use of 'monit' or 'systemd'

\section{Power Efficiency}

why and in what degree is power efficiency a requirement

Embedded systems have traditionally been the most power constrained computer systems. For deployment in urban areas these requirement is preserved but can be relaxed as it is easier to provide power on the system. In an urban environment provision can be made so that the system could switch to the town's electic distribution network when its battery or renewable resource are unsaficient. On the other hand, in rural areas power efficiency is much more important due to lack of an extensive power grid.

\subsection{Peripherals}

The choice of peripherals used has an impact on power consumption. As we show different usb-to-ethernet modules where consuming different amount of energy.

\begin{center}
 \begin{tabular}{| l | c |}
 \hline
   & Ampere \\
 \hline
 \hline
 Idle &  \\
 \hline
 + Ethernet1 & \\
 \hline
 + Ethernet2 & \\
 \hline
 + Keyboard & \\
 \hline
 \end{tabular}
\end{center}

Peripherals that are not necessary should be removed from the final deployment package (keyboard, UART)


\subsection{Local vs Remote processing}

In general the following can be assumed for cost of local versus remote processing of data.

\begin{center}
 \begin{tabular}{| l | c | c |}
 \hline
   & Processing Cost & Transfer Cost \\
 \hline
 \hline
 Local Processing & High & Low \\
 \hline
 Remote Processing & Low & High \\
 \hline
 \end{tabular}
\end{center}


A quick and low cost heuristic needs to be created in-order to decide if we will forward or process the raw sensor data.

\begin{lstlisting}
If (local_process_cost + processed_data_transfer_cost > raw_data_transfer_cost) {
  send(raw_data)
} else {
  process(raw_data)
  send(processed_data)
}
\end{lstlisting}

Another consideration especially when wireless communication is used, is the cost of dropped packages or retransmissions. In some cases the cost of retransmissions could surpass the power difference between local and remote processing of data. Thus the above algorithm should be augmented to represent that relation; possibly by multiplying the transfer cost with a retransmission factor.

\noindent
\textbf{Compression}

Use compression to reduce transmission cost. Can be beneficial under constraints.

\noindent
\textbf{Elastic Fidelity Computations}

Sensors are inherently faulty, thus imprecise computation can be employed to improve power efficiency


\subsection{Encryption}

Encryption is a process expensive process thus a power expensive one too. One approach to reduce power consumption of the encryption steps on the framework is to adopt symmetric encryption whenever possible. Also, different libraries could be profiled to find if there are significant benefits on choosing the one over the other.

\subsection{Kernel}

Kernel can be tuned to improve the power efficiency of the device. A first step would be to minimize the number of modules running at any point. Moreover, we can use a governor that has power-saving as a metric.

\textbf{WARNING: kernel 3.8 has a bug that causes reboot to fail when using the powersave governor. Use ``echo performance > /sys/devices/system/cpu/cpu0/cpufreq/scaling\_governor'' before reboot}


\begin{verbatim}
CPU Power Management  --->
 CPU Frequency scaling  --->
  Default CPUFreq governor (performance)  --->
   ( ) performance
   (X) powersave
   ( ) userspace
   ( ) ondemand
   ( ) conservative
\end{verbatim}


\subsection{The Buffer: buffer between the external and internal API}

The buffer in between the internal and external networks can be used to improve power efficiency. By choosing different policies for handling package forwarding we can reduce the consumed power. The reduced power comes at the cost of loss of accuracy.

The policies should be considered a second level techniques as initialy the polling frequency of the sensor should be altered; changing the polling frequency would provide even greater benefits. 

\subsubsection{Policies}

\begin{itemize}
 \item package drop
 \item sampling
 \item averaging
 \item compression
\end{itemize}


\appendix

\section{monit}

monit is a free, open source process supervision tool for Unix and Linux. \url{https://en.wikipedia.org/wiki/Monit}, \url{http://mmonit.com/monit/}

\url{http://mmonit.com/wiki/Monit/ConfigurationExamples}


\section{Collectl \& Colmux}

\subsection{Collectl}

From \url{http://collectl.sourceforge.net/}:  Unlike most monitoring tools that either focus on a small set of statistics, format their output in only one way, run either interatively or as a daemon but not both, collectl tries to do it all. You can choose to monitor any of a broad set of subsystems which currently include buddyinfo, cpu, disk, inodes, infiniband, lustre, memory, network, nfs, processes, quadrics, slabs, sockets and tcp. 

\subsubsection{Metrics}

\begin{itemize}
 \item Per CPU utilization
 \item Network traffic
\end{itemize}

\subsection{Colmux}

From  \url{http://collectl-utils.sourceforge.net/colmux.html} ``As its name implies, colmux is a collectl multiplexor, which allows one to collect data from multiple systems and treat it as a single data stream, essentially extending collectl's functionality to a set of hosts rather than a single one. Colmux has been tested on clusters of over 1000 nodes but one should also take note that this will put a heavier load on the system on which colmux is running. ``

\section{cpuset}
\tested{}


For more information see the following links:
\begin{itemize}
 \item \url{https://www.kernel.org/doc/Documentation/cgroups/cpusets.txt}
 \item \url{https://www.suse.com/documentation/slerte_11/slerte_tutorial/data/slerte_tutorial.html}
 \item \url{http://techpubs.sgi.com/library/tpl/cgi-bin/getdoc.cgi?coll=linux&db=bks&srch=&fname=/SGI_Admin/LX_Resource_AG/sgi_html/ch01.html}
\end{itemize}

\subsection{Install}

The cgroups are redesigned as of 2013, we may consider going in more recent versions of the Linux kernel, 3.15 or 3.16, to see what new things may be available.

\begin{verbatim}
General setup  --->
[*] Control Group support  --->
--- Control Group support
[ ]   Example debug cgroup subsystem
[ ]   Freezer cgroup subsystem
[*]   Device controller for cgroups                                                                 
[*]   Cpuset support                                                                                
[*]     Include legacy /proc/<pid>/cpuset file                                                      
[*]   Simple CPU accounting cgroup subsystem                                                        
[ ]   Resource counters
- -     Memory Resource Controller for Control Groups
- -       Memory Resource Controller Swap Extension
- -         Memory Resource Controller Swap Extension enabled by default
- -       Memory Resource Controller Kernel Memory accounting (EXPERIMENTAL)
- -     HugeTLB Resource Controller for Control Groups
[ ]   Enable perf\_event per-cpu per-container group (cgroup) monitoring
[*]   Group CPU scheduler  --->
[ ]   Block IO controller
- -     Enable Block IO controller debugging
\end{verbatim}


From man page: ``If a system supports cpusets, then it will have the entry \textit{nodev cpuset} in the file /proc/filesystems''.

\begin{lstlisting}
# cat /proc/filesystems | grep cpuset
\end{lstlisting}

\textbf{There is a way to restrict all process to the cpuset and it happens using bootcpuset. I could not find it, however it may be possible through the init= boot argument}


\subsubsection{User-level control application}
apt-get instal cpuset

\subsection{cgroups filesystem}
\url{https://access.redhat.com/documentation/en-US/Red_Hat_Enterprise_Linux/6/html/Resource_Management_Guide/sec-cpu.html}

\begin{lstlisting}
mount -t cgroup cgroup /sys/fs/cgroup
\end{lstlisting}

\subsection{Creating a Set}

\begin{lstlisting}
cset set --cpu=2-4 --mem=0 --set=<SET_NAME>
\end{lstlisting}


\subsection{Adding a Process to a Set}

\begin{lstlisting}
cset proc --set <SET_NAME> --exec -- <APP> <APP_ARGS> & 
\end{lstlisting}

\subsection{Upstart Init Configuration}


\section{Stress Testing}

\subsection{CPU}

\begin{tabular}{| l | c | c | c |}
\hline
 Stress Test Suite & Active Cores & CPU Utilization & Ampere \\
\hline
\hline
 U-Boot (pre-boot) & 0 & 0 & 0.248 \\
\hline
 Idle (post-boot)  & 0 & 0 & 0.330\\
\hline
 \multirow{4}{*}{stress-ng} & 1 & 100 & 0.440\\
  & 2 & 100 & 0.550  \\
  & 3 & 100 & 0.665  \\
  & 4 & 100 & 0.795  \\
\hline
 \multirow{4}{*}{stress-py} & 1 & 100 & 0.510\\
  & 2 & 100 & 0.705  \\
  & 3 & 100 & 0.940  \\
  & 4 & 100 & 1.230  \\
\hline
\end{tabular}

The difference between \textit{stress-ng}\cite{stress-ng} and \textit{stress-py}\cite{stress-py} is probably due to utilization of the floating point functional units with \textit{stress-py}.

\subsection{Network}

To stress the ethernet controller we are using \textit{iperf}\cite{iperf}. In all stress cases the CPU utilization reaches 100%. To generate different stress utilization we exercise the tcp window size from 1k to 20k.


\begin{tabular}{| l | c | c | c |}
\hline
 Cable & TCP window size & Ampere & CPU Utilization \\
\hline
\hline
 \multirow{6}{*}{Default} & 1k & 0.340 & \\
  & 2k & 0.410 & \\
  & 4k & 0.455 & \\
  & 8 & 0.495 & \\
  & 16 & 0.535 & \\
  & (auto) & 0.565 & \\
\hline
 \multirow{6}{*}{Cable 2} & 1k & & \\
  & 2k & & \\
  & 4k & & \\
  & 8 & & \\
  & 16 & & \\
  & (auto) & & \\
\hline
\end{tabular}


\subsection{Memory}

\section{Communication Resilience}
\tested{}

We will be using netem to emulate different network conditions. \textit{netem} ``provides Network Emulation functionality for testing protocols by emulating the properties of wide area networks.'' \cite{netem}

\noindent
Resources:
\begin{itemize}
 \item \url{http://www.linuxfoundation.org/collaborate/workgroups/networking/netem}
 \item \url{http://www.opensourceforu.com/2012/06/bandwidth-throttling-netem-network-emulation/}
 \item \url{https://calomel.org/network_loss_emulation.html}
\end{itemize}

\subsection{Install}

Enable the kernel module (version $>=$ 2.6)

\begin{verbatim}
Networking -->
 Networking Options -->
  QoS and/or fair queuing -->
   Network emulator
\end{verbatim}

\subsection{Use}

General options:
\begin{verbatim}
Usage: ... netem [ limit PACKETS ] 
                 [ delay TIME [ JITTER [CORRELATION]]]
                 [ distribution {uniform|normal|pareto|paretonormal} ]
                 [ drop PERCENT [CORRELATION]] 
                 [ corrupt PERCENT [CORRELATION]] 
                 [ duplicate PERCENT [CORRELATION]]
                 [ reorder PRECENT [CORRELATION] [ gap DISTANCE ]]
\end{verbatim}

\subsubsection{Examples}

\noindent
Delete queue discipline from interface:
\begin{lstlisting}
tc qdisc del dev eth0 root
\end{lstlisting}

\noindent
Change the loss rate on the queue discipline of an interface
\begin{lstlisting}
tc qdisc change dev eth0 root netem loss 0.1%
\end{lstlisting}

\noindent
Change the duplicate/corruption rate on the queue discipline of an interface
\begin{lstlisting}
tc qdisc change dev eth1 root netem duplicate/corrupt 1%
\end{lstlisting}

\noindent
Add a constant delay and jitter to the queue discipline of an interface
\begin{lstlisting}
tc qdisc add dev eth1 root netem delay 80ms 10ms
\end{lstlisting}


\section{Linux Kernel}

Resources:
\begin{itemize}
 \item [3.16] https://github.com/dsd/linux-odroid/
 \item [3.8] https://github.com/hardkernel/linux
\end{itemize}

\subsection{Compile}

\begin{lstlisting}
make ARCH=arm CROSS_COMPILE=arm-linux-gnueabi- 
make ARCH=arm CROSS_COMPILE=arm-linux-gnueabi- uImage modules exynos4412-smdk4412.dtb
\end{lstlisting}

\begin{lstlisting}
make ARCH=arm CROSS_COMPILE=arm-linux-gnueabi- odroidu2_defconfig
make ARCH=arm CROSS_COMPILE=arm-linux-gnueabi- menuconfig
<Enable Custom Options> ...
make ARCH=arm CROSS_COMPILE=arm-linux-gnueabi- zImage
make ARCH=arm CROSS_COMPILE=arm-linux-gnueabi- modules
LOADADDR=40008000 make ARCH=arm CROSS_COMPILE=arm-linux-gnueabi- uImage
\end{lstlisting}

\subsection{Install}

\begin{lstlisting}
cp arch/arm/boot/zImage /media/boot/
cp .config /media/rootfs/boot/config-`cat ./include/config/kernel.release`
make ARCH=arm CROSS_COMPILE=arm-linux-gnueabi- modules_install INSTALL_MOD_PATH=/media/rootfs/
make ARCH=arm CROSS_COMPILE=arm-linux-gnueabi- firmware_install INSTALL_FW_PATH=/media/rootfs/lib/firmware
\end{lstlisting}

\section{U-Boot}

Resources:
\begin{itemize}
 \item mainline
\end{itemize}

\subsection{Compile}

\begin{lstlisting}
make ARCH=arm CROSS_COMPILE=arm-linux-gnueabi- odroid_config
make ARCH=arm CROSS_COMPILE=arm-linux-gnueabi- -j 2 all
\end{lstlisting}

\subsection{Install}

Then we need to use 'u-boot-dtb.bin'.

\textbf{PUT EXAMPLE BOOT.SCR HERE}

%% ## Copy boot scripts

%% use
%% https://repo.anl-external.org/repos/cerisc/forest/branches/2014/getziadz/code/sdcardTools/dummyBoot/

%% copy the boot.* files 



\section{Upstart}

\url{http://fog.ccsf.edu/~gboyd/cs260a/online/startup/upstart.html}

\section{Creating an Image}

\begin{enumerate}
 \item Complile UBoot
 \item Compile Kernel
 \item Format SDCard
 \item Flush boot binaries to SDCard
 \item Get rootfs
 \item copy rootfs to SDCard
 \item Install Kernel, boot.scr, and btd to SDCard
 \item Configure /etc files
 \item Install extra packages
\end{enumerate}

\subsection{Format SDCard}

Using the following script type: \texttt{./mkcardOdroid.sh /dev/<sd\_card\_device>}
\lstinputlisting{data/mkcardOdroid.sh}

\subsubsection{Flush Bootloader binaries in SDCard}

Using the following script type: \texttt{./sd\_fusing.sh /dev/<sd\_card\_device>}
\lstinputlisting{data/sd_fusing.sh}

\subsection{Populate the rootfs partition}

There are two way to populate the rootfs partition: the fastest and easiest way is to download the rootfs provided by the linaro, the second way is to start from scratch and using live-build, bootstrap, or multistrap generate and configure a custom rootfs. The second option is laborious but provides significant flexibility and should be seriously considered. To get the latest rootfs from linaro use the following command:

\begin{lstlisting}
wget -r --no-parent -P latest http://snapshots.linaro.org/ubuntu/images/nano/latest/
\end{lstlisting}

and then uncompress and copy the content of the 'binary' folder to the rootfs partitions you generated on the SDCard.

\subsection{Linux Kernel}

See the above \textit{Kernel} subsection.

\subsection{U-Boot}

See above \textit{U-Boot} subsection.

\subsection{Configure /etc}

\subsubsection{Networking}

We need to edit /etc/network/interfaces so that ethernet is brought up at boot time and IPs are assigned from dhcp

\begin{lstlisting}
++ auto eth0 eth1
++ iface eth0 inet dhcp
++ iface eth1 inet dhcp
\end{lstlisting}

In addition a permanent solution should be found about the MAC problem. The MAC is autogenerated at first boot and saved into a file, this creates problems when using image cloning to generate new SDCards.


\subsubsection{Extra Packages}

On the first boot run the following command sequence to install required packages.
\begin{lstlisting}
apt-get update
# Enable remote logins
apt-get install openssh-server
# Required by waggle code
apt-get install python-pika 
# For editing
apt-get install nano
\end{lstlisting}

\noindent
\textit{After installing openssh follow the instructions on the security chapter to restrict its access}

\subsubsection{Other}

After the first boot a new password should be setup.

We may also create a new user that will be responsible for running the majority of the waggle code.

\section{Known Issues}

\subsection{Clock}

The clock resets after power lose.
\begin{itemize}
 \item \url{http://www.hardkernel.com/main/products/prdt_info.php?g_code=G137508214939}
 \item \url{http://forum.odroid.com/download/file.php?mode=view&id=92}
\end{itemize}

\subsection{MAC Address}

The MAC address is generated at first boot. If a single image is used to flash all nodes this will cause all nodes to have the same MAC address.


\section{The Node Controller Buffer}

Why we chose to have a buffer between the internal and external communication.

\subsection{Python Manager Implementation}

From \url{https://docs.python.org/2/library/multiprocessing.html#managers}: Managers provide a way to create data which can be shared between different processes. A manager object controls a server process which manages shared objects. Other processes can access the shared objects by using proxies.

\subsubsection{Advantages}

\subsubsection{Disadvantages}

Slower than shared memory.

\subsubsection{Tests}

\lstinputlisting{data/manager/server.py}
\lstinputlisting{data/manager/client.py}

\begin{tabular}{| l | c | c |}
\hline
\multicolumn{3}{|c|}{Memory Benchmark} \\
\hline
Number of 1024B Entries & VIRT &  RES\\
\hline
\hline
1M & \multicolumn{2}{|c|}{Out of Memory} \\
\hline
100K & 132M & 112M\\ 
\hline
10K & 38M & 17M\\ 
\hline
1K & 28M & 7M\\ 
\hline
\end{tabular}



\bibliography{main}{}
\bibliographystyle{plain}

\end{document}
