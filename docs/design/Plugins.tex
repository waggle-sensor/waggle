\subsection{Plugins}

\subsubsection{System plugins}

\begin{itemize}
\item {System\_send: } packetizes JSON data into Waggle message. Waggle messages are sent to attached parent node via TCP/IP.
\item {System\_receive: } receives Waggle messages, check CRC, de-serialize the message using JSON, and send it to system\_router plugin. 
\item {System\_router: } routes JSON data. This plugin holds routing table to distribute messages properly. For the details of the routing table, refer to 
\ref{routing}.
\end{itemize}

\subsubsection{User plugins}



The data types used in JSON are listed below.
\label{json_keywords}
\begin{center}
    \rowcolors{1}{rahmen!8}{white}
    \begin{tabular}{ | p{3cm} | p{3cm} | p{8cm} |}
    \hline
    \hline
    \textbf{KEYWORD} & \textbf{MEANING} & \textbf{NOTE} \\ \hline \hline
    msg\_mj\_type & Major operation & One char \\  \hline
    msg\_mi\_type & Minor operation & One char \\  \hline
    s\_puid & Plugin/Payload unique identifier & Sender's PUID  \\  \hline
    p\_puid & Plugin/Payload unique identifier & Recipient's PUID  \\  \hline
    meta & Meta data & Dictionary-type information (plugin name, version, instance, etc.)  \\  \hline
    data & Data & Dictionary-type data  \\  \hline
    rec & Recipient & Specify recipient (Default is Beehive) \\ \hline
    snd\_s & Sender's sequence number & Sequence number of the sender (optional) \\ \hline
    resp\_s & Responder's sequence number & Sequence number of the responder (optional) \\ \hline
    snd\_ss & Sender's session number & Session number of the sender (optional) \\ \hline
    resp\_ss & Responder's session number & Session number of the responder (optional) \\ \hline
    error & Error message & Error message when occur (optional) \\ \hline
    \end{tabular}
\end{center}

When JSON contains `error' plugins need to react to it appropriately.

\subsubsection{Communication}

All communications between system\_router and plugins are through JavaScript Object Notation (JSON).

\subsubsection{Registration}

A plugin should maintain a plugin unique identifier (PUID) generated by the nodecontroller the plugin attached to. If the plugin cannot find the PUID (either 
the first time or missing), the plugin can request a PUID by sending a registration request to the nodecontroller. As an example would be,

\begin{framed}
\noindent
\{``mj\_op'':``r'', ``mi\_op'':``r'', ``puid'':uuid.uuid4(), ``meta'':``... plugin name, version, instance...''\}
\end{framed}

For ``tmpID'' use randomly generated ID using uuid package in Python. This ``tmpID'' will only be used until the plugin gets PUID from the nodecontroller. The 
PUID should be properly maintained by the plugin (e.g., in a file) and should not be altered.

\subsubsection{Listener}

If a plugin needs to get data from outside (e.g., the Beehive server or nodes), the plugin should assign an asynchronous listener and make the listener 
attached to the incoming queue. For the format of incoming message, refer to \ref{json_keywords}.