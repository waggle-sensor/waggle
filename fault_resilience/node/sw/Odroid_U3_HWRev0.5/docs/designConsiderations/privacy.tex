\section{Privacy}

As mention earlier privacy becomes a first order considerations when deploying sensors on an urban environment. The reason for that is that as the sensors collect data, sensitive information about habitats or travelers around them may also be collected. This can happen accidentally or with a malicious intent. Following a list of possible sensors together with normal and abusive scenarios is gonna be presented.

The case of a wireless cards: a wireless card can act as an activity/traffic sensor. Metrics that can be recorded include traffic load, unique MAC addresses, possibly unique urls of unencrypted communication, or even keyword occurrence of unencrypted instant messaging. These information can be used to present an image of how communication heavy the area is, or how event news propagate from their source. However, miss-use can also happen and personal information can be extracted; like personal messages, preferences, or even walking routes and presence inside building.
  
The case of RFID readers: An RFID reader can be used for enabling some feature on the sensor or communication of the sensor and an application in an approaching device. In a malicious scenario the rfid reader can be used to extract information from rfid cards that come into its vicinity and otherwise have no correlation with the system (Ventra Cards, ID Cards, Credit Cards).
  
The case of image or video recording: image capturing can be used in conjunction with a figure recognition/tracking software to present information about the density of people or vehicles near the node. A possible miss-use would be face recognition for route tracking, presence verification, or even extraction of conversations.
  
The case of audio recording: audio recording can be used to collect information about noise levels or species identification in a location. Or it can be used to record conversations.
  
  
For the aforementioned reasons requirements about the operation and oversight of the sensors' data collection need to be specified.

\subsection{Operation Requirements}
Should we require some kind of key for a sensor to join the embedded system?


The embedded system should have control over what it senses all the time.
What is sensed should be public at all time and monitored so that in case of a privacy incident information leakage is minimized

Automatic isolation/lock of offending sensors.

Particularly intrusive sensors may need to be left out and alternatives should be considered.



\subsection{Oversight Requirements}

An independent, from the operational authority, observer should be able to have a full picture of what has been collected by the sensors and who has access to it in an easy manner.
